\documentclass{article}
%\usepackage[a4paper, total={6in, 8in}]{geometry}
\usepackage{geometry}
 \geometry{
 a4paper,
 total={210mm,297mm},
 left=20mm,
 right=20mm,
 top=-2mm,
 bottom=2mm,
 }
%\usepackage[margin=0.5in]{geometry}

\usepackage{amsmath,amssymb}
\usepackage{ifpdf}
%\usepackage{cite}
\usepackage{algorithmic}
\usepackage{array}
\usepackage{mdwmath}
\usepackage{pdfpages}
\usepackage{mdwtab}
\usepackage{eqparbox}
\usepackage{cite}
%\onecolumn
%\input{psfig}
\usepackage{color}
\usepackage{graphicx}
\setlength{\textheight}{23.5cm} \setlength{\topmargin}{-1.05cm}
\setlength{\textwidth}{6.5in} \setlength{\oddsidemargin}{-0.5cm}
\renewcommand{\baselinestretch}{1}
\pagenumbering{arabic}
\usepackage{ragged2e}
\renewcommand{\baselinestretch}{1.5}

\begin{document}

\textbf{
\begin{center}
{
\large{School of Engineering and Applied Science (SEAS), Ahmedabad University}\vspace{4mm}
}
\end{center}
%
\begin{center}
\large{B.Tech(CSE) Semester IV: Probability and Stochastic Processes (MAT 277) }\\ \vspace{3mm}
\end{center}
}
\begin{itemize}
\item Group No : BB14
\item Group Members :  %\item Roll no: s1749002 (Ph.d)
%\item Associated with Project: DST-UKIERI
\item Project Title: 

\end{itemize} 

\section {Justify how probabilistic model/PSP concept is used in your project. How uncertainty is modeled?}
{The objective of this project was to estimate the number of live births and fetal losses could occur in a given span of time. In order to do this, we have  used probability concepts like Binomial theorem, Multinomial theorem, probability distribution functions, cumulative distribution functions and Poisson Distribution.}
\begin{center}
	\Large Uncertainties
\end{center}

During the pregnancy, many uncertain events take place such as conception, mortality, fetal losses and number of births over a period of time,Occurrence of conception even after using contraceptives.\\

To find out the probability of conception, we modelled it as a Binomial Random Variable and.
We considered the probability of fetal losses to be fixed by deciding the mortality rate for the fetuses in order to find the pregnancy leading to Live Birth for different intervals of time . We also found the mean interval of months, that is the difference between two live births for the total period of 10,15 years etc.\\

Probability of conception in a fertile time follows geometric distribution\\

%% Insert AEZ Jinesh's part

We were able to model the probability of {\itshape r} numbers of live births given the probability of fetal mortality over the span of {\itshape t} years.

%Modeling of physical/real-time uncertain Problem, Study of any existing probability based models etc.,
%Include Block/State diagram (Optional)
%\item Explain the probabilistic model used to solve the problem

\section{Clearly enlist the new things done in the coding part, excluding the shared code. [If no new code is written/added/modified, then please write NA]}

\begin{enumerate}
    \item Mention code change-1
    \item Mention code change-2
\end{enumerate}



\section{Contribution of team members}	
\subsection{Technical contribution of all team members }
Enlist the technical contribution of members in the table. Redefine the tasks (e.g Task-1 as simulation of fig.1 and so on)
\begin{table}[h]
\centering
\begin{tabular}{|l|l|l|l|l|l|}
\hline
Tasks  & Team member 1 & Team member 2 & Team member 3 & Team member 4 & Team member 5 \\ \hline
Task-1 &               &               &               &               &               \\ \hline
Task-2 &               &               &               &               &               \\ \hline
Task-3 &               &               &               &               &               \\ \hline
\end{tabular}
\end{table}
\subsection{Non-Technical contribution of all team members }
Enlist the non-technical contribution of members in the table. Redefine the tasks (e.g Task-1 as report writing etc.)
\begin{table}[h]
\centering
\begin{tabular}{|l|l|l|l|l|l|}
\hline
Tasks  & Team member 1 & Team member 2 & Team member 3 & Team member 4 & Team member 5 \\ \hline
Task-1 &               &               &               &               &               \\ \hline
Task-2 &               &               &               &               &               \\ \hline
Task-3 &               &               &               &               &               \\ \hline
\end{tabular}
\end{table}

\section{Any innovation done considering the society/neighborhood problem?}

\section{Enumerate the inferences derived from user-centric perspective.}
	
\begin{enumerate}
\item Derived inference-1 from the work. 
\item Derived inference-2 from the work. 
\end{enumerate} 




\section*{Submission checklist for uploading on Google Drive}
This section provides the submission checklist for smooth and efficient submission process.  (This is for your reference and please remove this while writing your report).  
\begin{itemize}

\item Soft copy of this project Report
\item Soft copy of Concept Map 1
\item Soft copy of base article
%\item Hard copy of \textbf{turnitin report} (It should be less than 15 percent after excluding the bibliography)
\item Soft copy of hand written analysis (Optional)
%\item Softcopy of above four documents in pan-drive/hard drive.
\item Folder of matlab codes (with proper naming)
\item Folder of reproduced results and new results both in .fig and .jpg formats
%\item .pdf Scanned copy of analysis (handwritten)
\item latex (.tex) file of the project report.
\end{itemize}
%\vspace{0.5cm}




\bibliographystyle{IEEEtran}
\bibliography{ref.bib}

\end{document} 