\documentclass{article}
%\usepackage[a4paper, total={6in, 8in}]{geometry}
\usepackage{geometry}
 \geometry{
 a4paper,
 total={210mm,297mm},
 left=20mm,
 right=20mm,
 top=-2mm,
 bottom=2mm,
 }
%\usepackage[margin=0.5in]{geometry}

\usepackage{amsmath,amssymb}
\usepackage{ifpdf}
%\usepackage{cite}
\usepackage{algorithmic}
\usepackage{array}
\usepackage{mdwmath}
\usepackage{pdfpages}
\usepackage{mdwtab}
\usepackage{eqparbox}
\usepackage{cite}
%\onecolumn
%\input{psfig}
\usepackage{color}
\usepackage{graphicx}
\setlength{\textheight}{23.5cm} \setlength{\topmargin}{-1.05cm}
\setlength{\textwidth}{6.5in} \setlength{\oddsidemargin}{-0.5cm}
\renewcommand{\baselinestretch}{1}
\pagenumbering{arabic}
\usepackage{ragged2e}
\renewcommand{\baselinestretch}{1.5}

\begin{document}

\textbf{
\begin{center}
{
\large{School of Engineering and Applied Science (SEAS), Ahmedabad University}\vspace{4mm}
}
\end{center}
%
\begin{center}
\large{B.Tech(CSE) Semester IV: Probability and Stochastic Processes (MAT 277) }\\ \vspace{3mm}
\end{center}
}
\begin{itemize}
\item Group No : BB14
\item Group Members :  %\item Roll no: s1749002 (Ph.d)
%\item Associated with Project: DST-UKIERI
\item Project Title: 

\end{itemize} 

\section {Justify how probabilistic model/PSP concept is used in your project. How uncertainty is modeled?}
{The objective of this project was to estimate the number of live births and fetal losses could occur in a given span of time. In order to do this, we have  used probability concepts like Binomial theorem, Multinomial theorem, probability distribution functions, cumulative distribution functions and Poisson Distribution.}
\begin{center}
	\Large Uncertainties
\end{center}

During the pregnancy, many uncertain events take place such as conception, mortality, fetal losses and number of births over a period of time,Occurrence of conception even after using contraceptives.\\

To find out the probability of conception, we modelled it as a Binomial Random Variable and.
We considered the probability of fetal losses to be fixed by deciding the mortality rate for the fetuses in order to find the pregnancy leading to Live Birth for different intervals of time . We also found the mean interval of months, that is the difference between two live births for the total period of 10,15 years etc.\\

Probability of conception in a fertile time follows geometric distribution\\

%% Insert AEZ Jinesh's part

We were able to model the probability of {\itshape r} numbers of live births given the probability of fetal mortality over the span of {\itshape t} years.

%Modeling of physical/real-time uncertain Problem, Study of any existing probability based models etc.,
%Include Block/State diagram (Optional)
%\item Explain the probabilistic model used to solve the problem

\section{Clearly enlist the new things done in the coding part, excluding the shared code. [If no new code is written/added/modified, then please write NA]}

\begin{enumerate}
    \item Mention code change-1
    \item Mention code change-2
\end{enumerate}



\section{Contribution of team members}	
\subsection{Technical contribution of all team members }
Enlist the technical contribution of members in the table. Redefine the tasks (e.g Task-1 as simulation of fig.1 and so on)
\begin{table}[h]
\centering
\begin{tabular}{|l|l|l|l|l|l|l|l}
\hline
Tasks  & Jinesh Salot & Kathan Shah & Nipun Patel & Poojan Gandhi & Rohan Parikh & Samkit Kundalia & Tirth Patel \\ \hline
Task-1 &      &       &        &          &     &       &         \\ \hline
Task-2 &      &       &        &          &     &       &         \\ \hline
Task-3 &      &       &        &          &     &       &       \\ \hline
\end{tabular}
\end{table}
\subsection{Non-Technical contribution of all team members }
Enlist the non-technical contribution of members in the table. Redefine the tasks (e.g Task-1 as report writing etc.)
\begin{table}[h]
\begin{tabular}{|l|l|l|l|l|l|l|l}
	\hline
	Tasks  & Jinesh Salot & Kathan Shah & Nipun Patel & Poojan Gandhi & Rohan Parikh & Samkit Kundalia & Tirth Patel \\ \hline
	Task-1 &      &       &        &          &     &       &         \\ \hline
	Task-2 &      &       &        &          &     &       &         \\ \hline
	Task-3 &      &       &        &          &     &       &       \\ \hline
\end{tabular}
\end{table}

\section{Any innovation done considering the society/neighborhood problem?}

\section{Enumerate the inferences derived from user-centric perspective.}
We find the probability of conception using the length of the cycle, number of fertile days in that cycle, number of coital acts in that cycle as a factor. This will be useful to couples as using this they would be able to calculate the probability of conception when they use contraceptive.\\

We have calculated the mean number of months between two live births given the probability of conception (considering that the couple is using contraceptives) taking the mortality rate of fetuses as a factor. This would be useful to couples who use contraceptives and want to know the chances of the number of births of the child and also the time interval of birth between them even after using contraceptives.\\
\begin{table}[h]
\begin{tabular}{|c|c|}
	\hline
	Fetal Mortality Rate  & Mean number of months between two livebirths\\ \hline
	0\% & 114     \\ \hline
	10\% & 125.6 \\ \hline
	25\% & 148.7 \\ \hline
\end{tabular}
\end{table}



\begin{enumerate}
\item Even after using the contraceptive, couples should at least expect 1 child as the expected values of births are 1.5, 1.4 and 1.2 for probability of fetal mortality=0,0.1,0.25.
\item For couples who want to have a child there is a 20\% chance that over a 15 year period they will have 2 or 3 Stillbirths/Miscarriages.
\item By modelling the probabilities of r births in y years we can see that even using the contraceptive pills (thus reducing the probability of conception to 0.01) for a period of 10 years there is 0.6273 probability that at least 1 child is born.
\end{enumerate} 


\bibliographystyle{IEEEtran}
\bibliography{ref.bib}

\end{document} 