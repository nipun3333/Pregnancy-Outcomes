\documentclass{article}
%\usepackage[a4paper, total={6in, 8in}]{geometry}
\usepackage{geometry}
 \geometry{
 a4paper,
 total={210mm,297mm},
 left=20mm,
 right=20mm,
 top=-2mm,
 bottom=2mm,
 }
%\usepackage[margin=0.5in]{geometry}

\usepackage{amsmath,amssymb}
\usepackage{ifpdf}
%\usepackage{cite}
\usepackage{algorithmic}
\usepackage{array}
\usepackage{mdwmath}
\usepackage{pdfpages}
\usepackage{mdwtab}
\usepackage{eqparbox}
\usepackage{cite}
%\onecolumn
%\input{psfig}
\usepackage{color}
\usepackage{graphicx}
\setlength{\textheight}{23.5cm} \setlength{\topmargin}{-1.05cm}
\setlength{\textwidth}{6.5in} \setlength{\oddsidemargin}{-0.5cm}
\renewcommand{\baselinestretch}{1}
\pagenumbering{arabic}
\usepackage{ragged2e}
\renewcommand{\baselinestretch}{1.5}

\begin{document}

\textbf{
\begin{center}
{
\large{School of Engineering and Applied Science (SEAS), Ahmedabad University}\vspace{3mm}
}
\end{center}
%
\begin{center}
\large{B.Tech(CSE) Semester IV: Probability and Stochastic Processes (MAT 277) }\\ \vspace{2mm}
\end{center}
}
\begin{itemize}
\item Group No : BB14
\item Group Members : \\ Jinesh Salot (AU1940178), Kathan Shah (AU1940152), Nipun Patel (AU1940033), Poojan Gandhi (AU1940125), Rohan Parikh (AU1940157), Samkit Kundalia (AU1940021), Tirth Patel (AU1940137)
 %\item Roll no: s1749002 (Ph.d)
%\item Associated with Project: DST-UKIERI
\item Project Title: 

\end{itemize} 

\section {Justify how probabilistic model/PSP concept is used in your project. How uncertainty is modeled?}
{The aim of our project was to estimate the probability of at-least {\slshape r} live births in given span of time and the mean difference between two live births. Probability concepts used for our modelling are {\slshape Binomial Distribution, Geometric Distribution, Poisson Distribution, Probability Mass Function (PMF) and Cumulative Mass Function (CMF).}\\
Many uncertain events occur before, during and after pregnancy. To list a few - coital frequency, fertility period during menstrual cycle, chances of conception, fetal losses, number of births over a certain period of time.\\
The Coital Frequency is modelled using {\slshape Poisson Random Variable} and lambda will be the mean daily probability of intercourse in a month of 30 days. The fertile period is considered to be 4-5 days. The probability of fetal loss is considered to be fixed in order to find the pregnancy leading to live birth for different intervals of time (in years). \\
Probability of conception is assumed to be fixed in the base article but we have modelled it using Geometric Distribution,PMF and CMF as part of our innovation work. Using Binomial Distribution where probability of success is the probability of conception and using Cumulative distribution Function over all ranges of X and v where is X describes the event of a live birth and v describes the event of a fetal loss, we modelled the probability of at-least r live birth in a t given years over various mortality rates. Also our model estimates the mean difference between two live births in a span of different t years.}\\
\begin{center}
	{\large {\bfseries Uncertainties}}
\end{center}

Pregnancy outcomes are uncertain in nature, that depend on a lot of factors like: coital frequency,
susceptibilty,
probability of conception,
use of contraceptives,
probability of fetal loss,
fecundability and
length of menstrual cycle etc. %Occurrence of conception even after using contraceptives.\\

To ascertain the Conception Probability - we modelled it as a Binomial Random Variable.
We considered the probability of fetal losses to be fixed by deciding the mortality rate for the fetuses in order to find the pregnancy leading to Live Birth for different intervals of time . We also found the mean interval of months, that is the difference between two live births for the total period of 10,15 years etc.\\

%Probability of conception in a fertile time follows geometric distribution\\
%{\large insert jinesh's part}
%% Insert AEZ Jinesh's part

We were able to model the probability of {\itshape r} numbers of live births given the probability of fetal mortality over the span of {\itshape t} years.

%Modeling of physical/real-time uncertain Problem, Study of any existing probability based models etc.,
%Include Block/State diagram (Optional)
%\item Explain the probabilistic model used to solve the problem

\section{Clearly enlist the new things done in the coding part, excluding the shared code.}

\begin{enumerate}
    %\item Mention code change-1
    \item Modelled probability of conception which previously the author had taken constant. We did it using concepts of conditional probability and binomial random variables.
    \item Used coital frequency, number of fertile days, coital frequency between fertile days and days of intercourse occurred in fertile period as parameter to find probability of conception which was taken constant by author previously.
    \item Tuned the authors model to find the probability of exactly r fetal loss in a period of y years.
    %\item {\large TOBEDONE} %TO BE DONE AEZ 4th Point
\end{enumerate}

\section{Contribution of team members}	
\subsection{Technical contribution of all team members }
\begin{table}[h]
\centering
\begin{tabular}{|m{3.8154cm}|m{1.4154cm}|m{1.4154cm}|m{1.4154cm}|m{1.4154cm}|m{1.4154cm}|m{1.4154cm}|m{1.4154cm}|}
\hline
Tasks  & Jinesh Salot & Kathan Shah & Nipun Patel & Poojan Gandhi & Rohan Parikh & Samkit Kundalia & Tirth Patel \\ \hline
Recreating Results of Base Paper &  {\Large \checkmark}    & {\Large \checkmark}      & {\Large \checkmark}       & {\Large \checkmark}         & {\Large \checkmark}    &  {\Large \checkmark}     &  {\Large \checkmark}       \\ \hline
Coding for Innovation Part &    {\Large \checkmark}  &       &  {\Large \checkmark}      &   {\Large \checkmark}       &     &    & {\Large \checkmark}        \\ \hline
Integrating Innovation into the existing model &      &  {\Large \checkmark}     &        &   {\Large \checkmark}       & {\Large \checkmark}    &  {\Large \checkmark}     &  {\Large \checkmark}     \\ \hline
\end{tabular}
\end{table}

\subsection{Non-Technical contribution of all team members }
\begin{table}[h]
		\begin{tabular}{|p{3.8154cm}|p{1.4154cm}|p{1.4154cm}|p{1.4154cm}|p{1.4154cm}|p{1.4154cm}|p{1.4154cm}|p{1.4154cm}|}
		%\begin{tabular}{|l|l|l|l|l|l|l|l|}
			\hline
			Tasks  & Jinesh Salot & Kathan Shah & Nipun Patel & Poojan Gandhi & Rohan Parikh & Samkit Kundalia & Tirth Patel \\ \hline
			Research for Innovation &  {\Large \checkmark}    & {\Large \checkmark}      &{\Large \checkmark}        &  {\Large \checkmark}        & {\Large \checkmark}    &   {\Large \checkmark}    &  {\Large \checkmark}       \\ \hline
			Concept Map -1 &   {\Large \checkmark}   &  {\Large \checkmark}     &        &   {\Large \checkmark}       &     &      &  {\Large \checkmark}       \\ \hline
			Meeting with Medical Experts &      &  {\Large \checkmark}     &    {\Large \checkmark}    &          &   {\Large \checkmark}  &   {\Large \checkmark}    &       \\ \hline
			Report & {\Large \checkmark}     &       &   {\Large \checkmark}     &          & {\Large \checkmark}    &  {\Large \checkmark}     & {\Large \checkmark}      \\ \hline
		\end{tabular}
\end{table}


\section{Any innovation done considering the society/neighborhood problem?}
After consulting industry professionals for the project, we got to know their perspective and how serious the issue is. We also interacted with a couple who faced an undesirable pregnancy outcome. This is a very serious yet less talked about problem because of the sensitivity involved. Family planning is always something that is of vital importance to any couple.

We realised that in our case study, between the theoretical models and practical results - there were vast differences. Due to the complexity of many factors involved - the base paper assumed many of them to be constant. However, we wanted to try to minimize the gap between reality and theory as much as possible. We chose X as RV, (AEZYKMI)

We hope our effort to solve this problem serves as a pillar to future studies and groups in understanding it. With further research and understanding of interdisciplinary domains surrounding it - we believe that, our model can be of vital importance to the society.  


\section{Enumerate the inferences derived from user-centric perspective.}
We find the probability of conception using the length of the cycle, number of fertile days in that cycle and number of coital acts in that cycle as a factor. This will be useful to couples for family planning by calculating the probability of conception when they use contraceptives.\\
`
We have calculated the mean number of months between two live births given the probability of conception (considering that the couple is using contraceptives) taking the mortality rate of fetuses as a factor. This would be useful to couples who use contraceptives and want to know the chances of the number of births of the child and also the time interval of birth between them even after using contraceptives.\\
	\begin{table}[h]
		\begin{center}
			\begin{tabular}{|c|c|}
				\hline
				Fetal Mortality Rate  & Mean number of months between two livebirths\\ \hline
				0\% & 114     \\ \hline
				10\% & 125.6 \\ \hline
				25\% & 148.7 \\ \hline
			\end{tabular}
		\end{center}
	\end{table}


\begin{enumerate}
\item Even after using the contraceptive, couples should at least expect 1 child as the expected values of births are 1.5, 1.4 and 1.2 for probability of fetal mortality=0,0.1,0.25.
\item For couples who want to have a child there is a 20\% chance that over a 15 year period they will have 2 or 3 Stillbirths/Miscarriages.
\item By modelling the probabilities of r births in y years we can see that even using the contraceptive pills (thus reducing the probability of conception to 0.01) for a period of 10 years there is 0.6273 probability that at least 1 child is born.
\end{enumerate} 

\section{References}


\bibliographystyle{IEEEtran}
\bibliography{ref.bib}

\end{document} 